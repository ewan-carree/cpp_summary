\documentclass[a4paper, 12pt, titlepage]{scrartcl} %document type
\usepackage[utf8]{inputenc} %encoding
\usepackage[T1]{fontenc}    
\usepackage[english]{babel} %language

\usepackage{graphicx}

\titlehead{\centering\includegraphics[width=6cm]{cpp}}
\title{cpp\_summary}
\author{}
\date{August 31, 2020 - \today}
\publishers{Ewan Carree}

%manage cells in tables
\usepackage{makecell}

%manage colors
\usepackage{xcolor}
%\usepackage{color}

%New colors defined below
\definecolor{codegray}{rgb}{0.5,0.5,0.5}
\definecolor{backcolour}{rgb}{0.95,0.95,0.92}

% add code
\usepackage{listings}
%Code listing style named "mystyle"
\lstdefinestyle{mystyle}{
  backgroundcolor=\color{backcolour},   commentstyle=\color{codegray},
  keywordstyle=\color{magenta},
  numberstyle=\tiny\color{codegray},
  stringstyle=\color{black},
  basicstyle=\ttfamily\footnotesize,
  breakatwhitespace=false,         
  breaklines=true,                 
  captionpos=b,                    
  keepspaces=true,                 
  numbers=left,                    
  numbersep=5pt,                  
  showspaces=false,                
  showstringspaces=false,
  showtabs=false,                  
  tabsize=4
}
%"mystyle" code listing set
\lstset{style=mystyle}

%\usepackage[top=2cm, bottom=2cm, left=2cm, right=2cm]{geometry} 

%extern and intern links
\usepackage{hyperref} %links
\hypersetup{
    colorlinks=true,
    linkcolor=black,
    urlcolor=blue,
    linktoc=all
}

\begin{document}

\maketitle

\newpage
\section{C++ reserved words}
In cpp, there are some reserved words that you can't use in your program as variables, function name, class name, ...

\vspace{35mm}

\begin{table}[h]
\begin{center}
{\renewcommand{\arraystretch}{2} %space between lines%
{\setlength{\tabcolsep}{1.5cm} %space between columns%
\begin{tabular}{|l|c|r|}
  \hline
  \hyperref[]{} & \hyperref[]{} & \hyperref[]{} \\
  \hline
  \hyperref[]{} & \hyperref[]{} & \hyperref[]{} \\
  \hline
  \hyperref[]{} & \hyperref[]{} & \hyperref[]{} \\
  \hline
  \hyperref[]{} & \hyperref[]{} & \hyperref[]{} \\
  \hline
  \hyperref[]{} & \hyperref[]{} & \hyperref[]{} \\
  \hline
  \hyperref[]{} & \hyperref[]{} & \hyperref[]{} \\
  \hline
  \hyperref[]{} & \hyperref[]{} & \hyperref[]{} \\
  \hline
  \hyperref[]{} & \hyperref[]{} & \hyperref[]{} \\
  \hline 
  \hyperref[]{} & \hyperref[]{} & \hyperref[]{} \\
  \hline 
  \hyperref[]{} & \hyperref[]{} & \hyperref[]{} \\
  \hline
  \hyperref[]{} & \hyperref[]{} & \hyperref[]{} \\
  \hline
\end{tabular}}}
\end{center}
\caption{Reserved words}
\end{table} \vspace{5mm}



\clearpage
\section{Types}

\subsection{Bool}
\subsection{Char}
\subsection{Int}
\subsection{Unsigned int}
\subsection{Double}
\subsection{String}

\newpage
\section{Variable}

\subsection{Reference}

\subsection{Const}
The \textbf{const} keyword specifies that a variable's value is constant and tells the compiler to prevent the programmer from modifying it. \textbf{type const my\_variable}. \textit{const applies on the element on the left side if there is one, otherwise on the rigth}.

\begin{lstlisting}[language=C++]
type const my_variable;
\end{lstlisting} \vspace{5mm}

You can also use a precompiler method with \textbf{\#define} to specify that a variable is going to be constant the whole time. It's especially use for real world constant like \textit{pi}, \textit{e}, ...

\begin{lstlisting}[language=C++]
#define Pi 3.141592
\end{lstlisting} \vspace{5mm}


\subsection{Reference}
In c++ you can use reference to save memory and gain in efficiency. A reference on a variable is another variable that contains the memory address of the first variable. You can create a reference by using \textbf{\&}.

\vspace{5mm}

Here is an example of how to use reference :
\begin{lstlisting}[language=C++]
int my_variable = 10;
int & my_reference = my_variable;
std::cout << "Here is my_variable : " << my_variable << " and here is my_reference : " << my_reference << std::endl;

// Here is my_variable : 10 and here is my_reference : 10
\end{lstlisting} \vspace{5mm}


\newpage
\section{Function}
A function is a block of organized, reusable code that is used to perform a single, related action. Functions provide better modularity for your application and a high degree of code reusing.

In c++, a function is created based on this model :

\begin{lstlisting}[language=C++]
function_type
display(argument_type argument)
{
  //core
}
\end{lstlisting} \vspace{5mm}

\subsection{Default arguments}
Default arguments are used for arguments that don't necessarily need a value. We assign a default value that can be overloaded if needed. Default arguments are always declared after non-default arguments into function's head. 

\subsection{Templates}
Function templates are special functions that can operate with generic types. This allows us to create a function template whose functionality can be adapted to more than one type or class without repeating the entire code for each type.

\newpage
\section{Struct}

\newpage
\section{Class}
Classes are a way of bringing together data and functionality. Creating a new class creates a new type and so new instances of this type can be built.

\newpage
\section{Toolbox}

\subsection{Increment and decrement}
You can simply increment and decrement your variables by using \textbf{++} or \textbf{--}.

\subsection{Flows}
There are three different flows in c++, one for text that the user enter while the program is running, one for the classic output for the program and one for the error output if there is a problem inside the program while it's running.

\vspace{5mm}

Here is an example of how to use them properly :
\begin{lstlisting}[language=C++]
std::string my_variable;
std::cin >> my_variable;
std::cout << my_variable << std::endl;
std::cerr << "We've got a problem Houston" << std::endl;

// 10
// 10
// We've got a problem Houston
\end{lstlisting} \vspace{5mm}

\textbf{WARNING} : You have to handle spaces if you use \textit{cin} flow because it doesn't take in account the spaces.

The last code become :
\begin{lstlisting}[language=C++]
std::string my_variable;
std::getline(std::cin, my_variable);
std::cout << my_variable << std::endl;
std::cerr << "We've got a problem Houston" << std::endl;

// Hello, world!
// Hello, world!
// We've got a problem Houston
\end{lstlisting} \vspace{5mm}

\textbf{WARNING} : If you use std::cin then getline() in your code, it won't work. You have to use \textit{std::cin.ignore()} after each std::cin you might use.


\subsection{Files}

\end{document}
